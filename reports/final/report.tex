\documentclass{report}


\usepackage{graphicx}
\usepackage{hyperref}
\usepackage{nth}
\usepackage{amsmath}
\usepackage{longtable}
\usepackage[margin=0.75in]{geometry}

\graphicspath{{./assets/}}
\title{SER-210 Final: Final Report}
\author{Thomas Kwashnak}


\begin{document}
\maketitle
\tableofcontents
\newpage

\chapter{Revised Requirements}

\section{Original User Stories}
\begin{itemize}
    \item As a user, I can sign in with my GitHub account to sign into the app
    \item As a user, I can view a list of recent chat rooms to quickly get back to what I was working on
    \item As a user, I can tap a star on a chat room to pin it to the top of the list so I have easy access to chat rooms I want
    \item As a user, I can select a chat room to open it up
    \item As a user, I can type and send a message into the chat room to interact with the conversation
    \item As a user, I can see chat messages appear so I can keep up with the conversation
    \item As a user, I can share a chat so I can invite friends into the chat
    \item As a user, I can create a new chat room for a chatRepository that I own so I can coordinate with other contributors
    \item As a user, I can type \# to reference a pull request or issue so viewers can click and navigate to that pull request or issue
    \item As a user, I can leave a chat room so I am not cluttered with chats I don't participate in
    \item As a user, I can view the app info so I know the app version and other information regarding the app
    \item As a user, I can log out of my account so I can log into another account
    \item As a user, I can click a link to redirect to the chatRepository page from the chat info
\end{itemize}
\section{Added User Stoires}
\begin{itemize}
    \item As a user, I can click on a linked pull request or issue to open up that item on the github website
\end{itemize}
\section{Modified User Stories}
\begin{itemize}
    \item As a user, I can view a list of recent chat rooms to quickly get back to what I was working on \begin{itemize}
        \item The chat rooms do not sort by "most recently updated". This was a minor detail in the user story that did not get implemented, but it was not an outstanding feature
    \end{itemize}
    \item As a user, I can log out of my account so I can log into another account\begin{itemize}
        \item While this may imply that chat rooms are account specific, they are actually device specific.
    \end{itemize}
\end{itemize}
\section{Finalized User Stories}
All user stories in this list has been completed
\begin{itemize}
    \item As a user, I can sign in with my GitHub account to sign into the app
    \item As a user, I can view a list of chat rooms to quickly get back to what I was working on
    \item As a user, I can tap a star on a chat room to pin it to the top of the list so I have easy access to chat rooms I want
    \item As a user, I can select a chat room to open it up
    \item As a user, I can type and send a message into the chat room to interact with the conversation
    \item As a user, I can see chat messages appear so I can keep up with the conversation
    \item As a user, I can share a chat so I can invite friends into the chat
    \item As a user, I can create a new chat room for a chatRepository that I own so I can coordinate with other contributors
    \item As a user, I can type \# to reference a pull request or issue so viewers can click and navigate to that pull request or issue
    \item As a user, I can leave a chat room so I am not cluttered with chats I don't participate in
    \item As a user, I can view the app info so I know the app version and other information regarding the app
    \item As a user, I can log out of my Github Account so I can log into another account
    \item As a user, I can click a link to redirect to the chatRepository page from the chat info
    \item As a user, I can click on a linked pull request or issue to open up that item on the github website

\end{itemize}

\chapter{Project Report}
\section{Requirement Completion}
In terms of requirements, I was able to get to everything major that I had planned on completing through the user stories. There were some after-thoughts or implied functionality that I was not able to get to, but they were more minor aspects. Firstly, I had planned to get the home screen to sort chat rooms by their recent messages, but I found that doing so would require fetching the entire data-set, and I could not think of an efficient way to manage that aspect. Additionally, I had planned to make a "pop-up" field that would show you what would be linked in the message that you sent, but I was unable to get to that with the time I was given.

\section{Project Execution}
Executing the project as a solo developer was both a blessing and a curse. I feel like I may have taken a bit too long to get started with each week's implementation, but I found that I was able to try things and freely figure it out without needing to keep everyone on the same page.

I spent most of the first 1-2 weeks focusing on the back end. The first few tasks I had to figure out was "how will I authenticate with GitHub accounts" and "how will I communicate and manage the firebase database server". After I figured out how integrating Firebase would work, I was then able to focus on the rest of the structure, which slowly fell into place once the Firebase situation had been resolved. Before knowing how Firebase would work, I was well equipped to have a datatable in the SQLite database for messages, but I realized that Firebase's api had all that covered way better than I could expect.

Once the back-end was complete, I found the front-end to just fall into place. Thanks to my foresight to set up the back-end with the front-end in mind, building the front end simply consisted of creating the interfaces and making the requests to the GithubWrapper and DatabaseWrapper

If I could start this project from the beginning, I don't think the overall process would change. I might try a better class structure, and for sure do things earlier on instead of waiting for the last few days, but overall nothing would change that much.


\end{document}
